\def\homeworkname{Copying Model}
\documentclass[assignment = 9]{homework}

\usepackage{caption, subcaption, pdfpages, float}
\usepackage{graphics, wrapfig, pgf, graphicx}
\usepackage{enumitem}


% pacotes para importar código
\usepackage{caption, booktabs}
\usepackage[inkscapepath=build/inkscape]{svg}
\usepackage[section, newfloat, outputdir=build/]{minted}
\definecolor{sepia}{RGB}{252,246,226}
\setminted{
    bgcolor = sepia,
    style   = pastie,
    frame   = leftline,
    autogobble,
    samepage,
    python3,
    breaklines
}
\setmintedinline{
    bgcolor={}
}

% ambientes de códigos de Python
\newmintedfile[pyinclude]{python3}{}
\newmintinline[pyline]{python3}{}
\newcommand{\pyref}[2]{\href{#1}{\texttt{#2}}}

% \SetupFloatingEnvironment{listing}{name=Código}
% \captionsetup[listing]{position=below,skip=-1pt}

\usepackage{csquotes}
\usepackage[
    style    = verbose-ibid,
    autocite = footnote,
    notetype = foot+end,
    backend  = biber
]{biblatex}
\addbibresource{references.bib}
\usepackage[section]{placeins}

\usepackage[hidelinks]{hyperref}
\usepackage[noabbrev, nameinlink]{cleveref}
\hypersetup{
    pdftitle  = {MO412/MC908 - RA187679},
    pdfauthor = {Tiago de Paula}
}

\newcommand{\textref}[2]{
    \hyperref[#2]{#1 \ref*{#2}}
}

\renewcommand{\vec}[1]{\mathbf{#1}}

\DeclareMathOperator{\round}{round}

\usepackage{import, multirow}
\usepackage{pgf, tikz}
\usetikzlibrary{matrix}
\usetikzlibrary{positioning}
\usetikzlibrary{automata}
\usetikzlibrary{shapes}

\usepackage{wrapfig}
\usepackage{booktabs}
\usepackage{amsmath, amssymb, bm, mathtools}
\usepackage{etoolbox, xpatch, xspace}
% \usepackage[mathcal]{euscript}
% \usepackage[scr]{rsfso}
% \usepackage{mathptmx}
\usepackage{relsize, centernot}
\usepackage{physics}


\makeatletter

% %% Símbolo QED %%
% \renewcommand{\qedsymbol}{\ensuremath{\mathsmaller\blacksquare}}

%% Marcadores de Prova: \direto, \inverso %%
\newcommand{\direto}[1][~]{\ensuremath{(\rightarrow)}#1\xspace}
\newcommand{\inverso}[1][~]{\ensuremath{(\leftarrow)}#1\xspace}

\undef\sum
%% Somatório: \sum_i^j, \bigsum_i^j %%
\DeclareSymbolFont{cmex10}{OMX}{cmex}{m}{n}
\DeclareMathSymbol{\sum@d}{\mathop}{cmex10}{"58}
\DeclareMathSymbol{\sum@t}{\mathop}{cmex10}{"50}
\DeclareMathOperator*{\sum}{\mathchoice{\sum@d}{\sum@t}{\sum@t}{\sum@t}}
\DeclareMathOperator*{\bigsum}{\mathlarger{\mathlarger{\sum@d}}}

%% Operadores de Conjunto: \pow(S), \Dom(S), \Img(S) %%
\DeclareSymbolFont{boondox}{U}{BOONDOX-cal}{m}{n}
\DeclareMathSymbol{\pow}{\mathalpha}{boondox}{"50}
\DeclareMathOperator{\Dom}{Dom}
\DeclareMathOperator{\Img}{Im}

\undef\Phi
%% Variantes gregas %%
\DeclareSymbolFont{cmr10}{OT1}{cmr}{m}{n}
\DeclareSymbolFont{cmmi10}{OML}{cmm}{m}{it}
\DeclareMathSymbol{\Phi}{\mathalpha}{cmr10}{"08}
\DeclareMathSymbol{\varpsi}{\mathalpha}{cmmi10}{"20}
\DeclareMathSymbol{\varomega}{\mathalpha}{cmmi10}{"21}

%% Família de Conjuntos: \family{S} %%
\DeclareMathAlphabet{\family}{OMS}{cmsy}{m}{n}

\undef\natural
\undef\real
%% Conjuntos Padrões: R, N, Z, C, Q %%
\DeclareMathOperator{\real}{\mathbb{R}}
\DeclareMathOperator{\natural}{\mathbb{N}}
\DeclareMathOperator{\integer}{\mathbb{Z}}
\DeclareMathOperator{\complex}{\mathbb{C}}
\DeclareMathOperator{\rational}{\mathbb{Q}}

%% Definição de Conjuntos: \set{ _ \mid _ } %%
\newcommand{\set}[1]{%
    \begingroup%
        \def\mid{\;\middle|\;}%
        \left\{#1\right\}
    \endgroup%
}

%% Novos Operadores: \modulo, \symdif, \grau %%
\DeclareMathOperator{\modulo}{~mod~}
\DeclareMathOperator{\symdif}{\mathrel{\triangle}}
\DeclareMathOperator{\grau}{deg}

%% Operadores Delimitados: \abs{\sum_i^j}, x \equiv y \emod{n} %%
% \newcommand{\abs}[1]{{\left\lvert\,#1\,\right\rvert}}
\newcommand{\emod}[1]{\ \left(\mathrm{mod}\ #1\right)}

%% Vérices e Arestas %%
\DeclareMathOperator{\Adj}{Adj}
\DeclareMathOperator{\dist}{dist}

\makeatother


\newenvironment{kmatrix}[1][1.3cm]{
    \begin{tikzpicture}[node distance=0cm]
        \tikzset{square matrix/.style={
                matrix of nodes,
                column sep=-\pgflinewidth, row sep=-\pgflinewidth,
                nodes={draw,
                    minimum height=#1,
                    anchor=center,
                    text width=#1,
                    align=center,
                    inner sep=0pt
                },
            },
            square matrix/.default=#1
        }
}{
    \end{tikzpicture}%
}

\newcommand*{\Scale}[2][4]{\scalebox{#1}{\ensuremath{#2}}}%

\newcommand{\red}[1]{\textcolor{red}{\textbf{#1}}}
\def\qm{?}

\begin{document}
    \pagestyle{main}

    Use the rate equation approach to show that the directed copying model (\href{http://networksciencebook.com/chapter/5#origins}{Section 5.9 of the book}) leads to a scale-free network with incoming degree exponent $\gamma_{\text{in}} = (2-p)/(1-p)$, where $p$ is the probability involved in the model.

    \section{Discrete Rate Equation}

We can see the preferential attachment showing up in the probability $\Pi(k)$ that a specific node $v$ with $\deg_\text{in}(v) = k$ will be connected to the new node:

\begin{align*}
    \Pi(k) &= p \cdot \mathbf{P}\left[\text{$v$ is chosen as the \textit{Target}}\right] + (1-p) \cdot \mathbf{P}\left[\text{$v$ is chosen as a \textit{Copied Connection}}\right] \\
    &= p \cdot \frac{1}{N} + (1-p) \cdot \frac{\deg_\text{in}(v)}{L} \\
    &= \frac{p}{N} + \frac{1-p}{L} k
\end{align*}

With this probability, we can analyze the graph generated by the directed version of the copying model.

\subsection{Growth at Degree \texorpdfstring{$\bm{k}$}{k}}

    For simplicity, we assume an initial model of a single loop, with $N = L = 1$. At each step, we add a new node and a new link, resulting in $N = L = t$, for any time step $t$. Let $N(k,t) = N \cdot p_{k,t}$ be the expected number of vertices with degree $k$ at time step $t$. Therefore, the expected number of new links to degree $k$ nodes is given by
    \begin{align*}
        R(k,t) &= \Pi(k) \cdot N(k,t) \\
            &= \left(\frac{p}{N} + \frac{1-p}{L} \cdot k\right) \cdot p_{k,t} N \\
            &= \left(\frac{p + \left(1 - p\right) k}{t}\right) p_{k,t} \cdot t \\
            &= \left(p + k - k p\right) p_{k,t}
    \end{align*}

    We can now model the growth of nodes of degree $k$ with the time-dependent rate equation:
    \begin{align*}
        N(k,t+1) &= N(k,t) + R(k-1,t) - R(k,t) \\
        (N + 1) p_{k,t+1} &= N p_{k,t} + (p + k-1 - (k-1) p) p_{k-1,t} - (p + k - k p) p_{k,t} \\
        N p_{k,t+1} + p_{k,t+1} &= N p_{k,t} + (2p + k - 1 - k p) p_{k-1,t} - (p + k - k p) p_{k,t}
    \end{align*}

\subsection{Stable Rate Equation}

    Finally, we can assume that $p_{k,t}$ will eventually stabilize as $p_k = \lim_{t \to \infty} p_{k,t}$. Therefore, when $t \to \infty$, the rate equation becomes
    \begin{align*}
        N p_k + p_k &= N p_k + (2p + k - 1 - k p) p_{k-1} - (p + k - k p) p_k \\
        0 &= -p_k + 2p p_{k-1} + k p_{k-1} - p_{k-1} - k p p_{k-1} - p p_k - k p_k + k p p_k
    \end{align*}

    Which can be rewritten as
    \begin{align*}
        p_k - 2p p_{k-1} + p_{k-1} + p p_k &= k p p_k - k p p_{k-1} - k p_{k} + k p_{k-1} \\
        p_k - 2 p p_{k-1} + p_{k-1} + p p_k &= k p (p_k - p_{k-1}) - k (p_k - p_{k-1}) \\
        p_k - p p_{k-1} + p_{k-1} &= (k p - k) (p_k - p_{k-1}) - p p_k + p p_{k-1} \\
        p_k - p p_{k-1} + p_{k-1} &= (k p - k) (p_k - p_{k-1}) - p (p_k - p_{k-1}) \\
        - p p_{k-1} + 2 p_{k-1} &= (k p - k - p) (p_k - p_{k-1}) - (p_k - p_{k-1}) \\
        (2 - p) p_{k-1} &= (k p - k - p - 1) (p_k - p_{k-1}) \\
        (p - 2) p_{k-1} &= (k + p + 1 - k p) (p_k - p_{k-1})
        \addtocounter{equation}{1}\tag{\theequation} \label{eq:stable}
    \end{align*}

    \section{Continuum Approximation}

For a large $k$, we can assume $p_k$ is continuous, such that $p_k \approx p_{k-1}$ and \[
    \dv{p_k}{k} \approx \frac{p_k - p_{k-1}}{k - (k-1)} = p_k - p_{k-1}
\]

Therefore, \cref{eq:stable} can be approximated by the following differential equation
\begin{align*}
    (p - 2) p_{k-1} \approx (p - 2) p_k &= (k + p + 1 - k p) \dv{p_k}{k} \\
        \frac{p - 2}{k + p + 1 - k p} &= \frac{1}{p_k} \dv{p_k}{k} \\
        - (2 - p) \int \frac{1}{(1 - p) k + p + 1} \dd{k} &= \int \frac{1}{p_k} \dv{p_k}{k} \dd{k} \\
        - \frac{2 - p}{1 - p} \ln\left(k + p + 1 - k p\right) + C_2 &= \ln p_k + C_1
\end{align*}

So,
\begin{align*}
    \ln p_k &= - \frac{2 - p}{1 - p} \ln\left(k + p + 1 - k p\right) + C_3 \\
    p_k &= C_4 \cdot \left(k + p + 1 - k p\right)^{-\frac{2 - p}{1 - p}}
\end{align*}

\subsection{The Degree Exponent}

    For $k \geq 1$, we have \[
        (1 - p) k \leq k + p + 1 - k p \leq k + 1 \leq 2 k
    \]

    And \[
        C_4 \cdot 2^{-\frac{2 - p}{1 - p}} \cdot k^{-\frac{2 - p}{1 - p}} \leq p_k \leq C_4 \cdot (1 - p)^{-\frac{2 - p}{1 - p}} \cdot k^{-\frac{2 - p}{1 - p}}
    \]

    Therefore,
    \[
        p_k \sim k^{-\frac{2 - p}{1 - p}} = k^{-\gamma_\text{in}}
    \]

    For $\gamma_\text{in} = \frac{2 - p}{1 - p}$, as proposed.


\end{document}
