\section{Discrete Rate Equation}

We can see the preferential attachment showing up in the probability $\Pi(k)$ that a specific node $v$ with $\deg_\text{in}(v) = k$ will be connected to the new node:

\begin{align*}
    \Pi(k) &= p \cdot \mathbf{P}\left[\text{$v$ is chosen as the \textit{Target}}\right] + (1-p) \cdot \mathbf{P}\left[\text{$v$ is chosen as a \textit{Copied Connection}}\right] \\
    &= p \cdot \frac{1}{N} + (1-p) \cdot \frac{\deg_\text{in}(v)}{L} \\
    &= \frac{p}{N} + \frac{1-p}{L} k
\end{align*}

With this probability, we can analyze the graph generated by the directed version of the copying model.

\subsection{Growth at Degree \texorpdfstring{$\bm{k}$}{k}}

    For simplicity, we assume an initial model of a single loop, with $N = L = 1$. At each step, we add a new node and a new link, resulting in $N = L = t$, for any time step $t$. Let $N(k,t) = N \cdot p_{k,t}$ be the expected number of vertices with degree $k$ at time step $t$. Therefore, the expected number of new links to degree $k$ nodes is given by
    \begin{align*}
        R(k,t) &= \Pi(k) \cdot N(k,t) \\
            &= \left(\frac{p}{N} + \frac{1-p}{L} \cdot k\right) \cdot p_{k,t} N \\
            &= \left(\frac{p + \left(1 - p\right) k}{t}\right) p_{k,t} \cdot t \\
            &= \left(p + k - k p\right) p_{k,t}
    \end{align*}

    We can now model the growth of nodes of degree $k$ with the time-dependent rate equation:
    \begin{align*}
        N(k,t+1) &= N(k,t) + R(k-1,t) - R(k,t) \\
        (N + 1) p_{k,t+1} &= N p_{k,t} + (p + k-1 - (k-1) p) p_{k-1,t} - (p + k - k p) p_{k,t} \\
        N p_{k,t+1} + p_{k,t+1} &= N p_{k,t} + (2p + k - 1 - k p) p_{k-1,t} - (p + k - k p) p_{k,t}
    \end{align*}

\subsection{Stable Rate Equation}

    Finally, we can assume the $p_{k,t}$ will eventually stabilize as $p_k = \lim_{t \to \infty} p_{k,t}$. Therefore, when $t \to \infty$, the rate equation becomes
    \begin{align*}
        N p_k + p_k &= N p_k + (2p + k - 1 - k p) p_{k-1} - (p + k - k p) p_k \\
        0 &= -p_k + 2p p_{k-1} + k p_{k-1} - p_{k-1} - k p p_{k-1} - p p_k - k p_k + k p p_k
    \end{align*}

    Which can be rewritten as
    \begin{align*}
        p_k - 2p p_{k-1} + p_{k-1} + p p_k &= k p p_k - k p p_{k-1} - k p_{k} + k p_{k-1} \\
        p_k - 2 p p_{k-1} + p_{k-1} + p p_k &= k p (p_k - p_{k-1}) - k (p_k - p_{k-1}) \\
        p_k - p p_{k-1} + p_{k-1} &= (k p - k) (p_k - p_{k-1}) - p p_k + p p_{k-1} \\
        p_k - p p_{k-1} + p_{k-1} &= (k p - k) (p_k - p_{k-1}) - p (p_k - p_{k-1}) \\
        - p p_{k-1} + 2 p_{k-1} &= (k p - k - p) (p_k - p_{k-1}) - (p_k - p_{k-1}) \\
        (2 - p) p_{k-1} &= (k p - k - p - 1) (p_k - p_{k-1}) \\
        (p - 2) p_{k-1} &= (k + p + 1 - k p) (p_k - p_{k-1})
        \addtocounter{equation}{1}\tag{\theequation} \label{eq:stable}
    \end{align*}
